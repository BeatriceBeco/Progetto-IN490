\documentclass[a4paper, twoside]{article}
\usepackage[T1]{fontenc}
\usepackage[utf8]{inputenc}
\usepackage[italian]{babel}
\usepackage[a4paper,heightrounded]{geometry}
	\geometry{outer=5cm}
	\geometry{inner=5cm}
\usepackage{amsmath,amssymb,amsthm}
\usepackage{microtype}
\usepackage{graphicx}
\usepackage[italiano,algoruled,lined,linesnumbered,shortend]{algorithm2e}
\usepackage{hyperref}
\theoremstyle{plain}
	\newtheorem{lemma}{Lemma}
	\newtheorem{prop}{Proposizione}
	\newtheorem{teor}{Teorema}
\theoremstyle{definition}
	\newtheorem*{eg}{Esempio}
	\newtheorem*{oss}{Osservazione}

\newcommand{\fall}{\ \forall}
\newcommand{\norma}[1]{\Vert#1\Vert}
\newcommand{\then}{\Rightarrow}
\newcommand{\vuoto}{\emptyset}

\newcommand{\N}{\mathbb{N}}
\newcommand{\Z}{\mathbb{Z}}
\newcommand{\Q}{\mathbb{Q}}
\newcommand{\R}{\mathbb{R}}
\newcommand{\C}{\mathbb{C}}

\newcommand{\U}{\mathcal U}

\newcommand{\Lim}{\lim\limits}
\newcommand{\Prod}{\prod\limits}
\newcommand{\Sum}{\sum\limits}
\newcommand{\Limite}[2]{\lim\limits_{#1\rightarrow#2}}

\newcommand{\al}{\alpha}
\newcommand{\be}{\beta}
\newcommand{\ga}{\gamma}
\newcommand{\de}{\delta}
\newcommand{\ve}{\varepsilon}
\newcommand{\vt}{\theta}
\newcommand{\la}{\lambda}
\newcommand{\mi}{\mu}
\newcommand{\si}{\sigma}
\newcommand{\vf}{\varphi}

\newcommand{\balgo}[2]
	{\SetKwInOut{Input}{Input}\SetKwInOut{Output}{Output}
		\SetNlSkip{1em}
		\BlankLine
		\Input{#1}
		\Output{#2}
		\BlankLine
		\DontPrintSemicolon
	}

\hypersetup{hidelinks}

\begin{document}
\begin{center}
	\textbf{Corso di Laurea in Matematica - Anno Accademico 2017-2018}

	\textbf{\Large IN490 - Linguaggi di Programmazione}

	\textsc{\small Docente: Marco Pedicini}

	\textsc{\small Studente: Daniele Salierno}
\end{center}
\section{Introduzione}
Lo scopo finale di questo progetto è la costruzione e l'attacco di un crittosistema di Elgamal e di un RSA costruiti a partire da primi di Sophie Germain.

Il documento si articola descrivendo gli algoritmi necessari al funzionamento del programma finale, seguendo il flusso presentato nel procedimento \ref{algo:workflow}, che sintetizza al massimo il lavoro da svolgere.

	\begin{algorithm}
	\BlankLine
	\DontPrintSemicolon
 	\caption{Workflow}
 	\label{algo:workflow}

 	Generazione di due primi di Sophie Germain $p<q$ sfruttando il Crivello ed il test di Solovay Strassen.\;
 	Creazione di un crittosistema di Elgamal con il primo $p$ trovato.\;
 	Creazione di un crittosistema RSA con i primi $p,q$ trovati.\;
 	Attacchi a Elgamal (Shanks e Pohlig Hellmann) e all'RSA (Wiener e $p-1$ o $\rho$ di Pollard).\;
	\end{algorithm}

	Una prima parte del programma si occuperà di generare un primo di Sophie Germain, sfruttando tecniche di crivello per scartare a priori alcuni numeri che non soddisfano certe restrizioni modulari (come vedremo nella sezione \ref{sec_sophie_germain}); i numeri che superano il crivello sono sottoposti al test di primalità di Solovay Strassen.

	Con i primi restituiti dall'algoritmo \ref{algo:sophie_germain} vengono costruiti e attaccati i due crittosistemi. Sappiamo che i primi di Sophie Germain sono numeri molto buoni per la costruzione di crittosistemi, quindi ci aspettiamo che gli attacchi falliscano o che impieghino comunque molto tempo per terminare con successo.

	\section{Generazione di Primi}\label{sec_sophie_germain}
		La prima fase del programma procede con due modalità: in un primo momento vengono generati i primi inferiori ad un bound B preso in input con il metodo del crivello di Eratostene; successivamente il programma passa alla modalità generazione di Sophie Germain, che è il cuore di questa sezione.

		Il counter salta diversi ordini di grandezza e comincia a restituire numeri molto grandi di cui verificare la primalità. Ciascun candidato deve superare i normali test di divisibilità per i primi piccoli trovati dal crivello e di test definiti in una nuova classe (estensione del vecchio filter di erathostenes): la ragion d'essere dei nuovi test è che non stiamo semplicemente cercando un numero primo, ma un primo di Sophie Germain, quindi sfruttiamo una condizione necessaria ma non sufficiente per scartarne alcuni e risparmiare tempo.
		\begin{prop}
			Se $p\equiv \frac{n-1} 2\mod n$ allora $p$ non è di Sophie Germain.
		\end{prop}
		\begin{proof}
			Se $p$ fosse di Sophie Germain, allora $q=2p+1$ sarebbe primo, ma
			\[2p+1=2\frac{n-1}{2}+1=n\equiv 0\mod n\]
			che contraddice l'ipotesi di primalità di $q$.$\dagger$
		\end{proof}

		\begin{algorithm}
		\BlankLine
		\DontPrintSemicolon
	 	\caption{Generazione di Primi di Sophie Germain}
	 	\label{algo:sophie_germain}

	 	Crivello di Eratostene con bound B.\;
	 	Ricerca di un numero $p$ maggiore di un bound M che soddisfi le restrizioni modulari e di divisibilità $p\neq \frac{f-1} 2\mod f\ \forall f$ filtro.\;
	 	Test di Solovay Strassen per determinare se $p$ è primo.\;
	 	Test di Solovay Strassen per determinare se $2p+1$ è primo, dunque se $p$ è di Sophie Germain.\;
		\end{algorithm}

		Quando un numero $p$ supera tutti i filtri è un candidato primo di Sophie Germain. Per prima cosa applichiamo il test di Solovay Strassen (algoritmo \ref{algo:solovay_strassen}) a $p$ effettuando 30 iterazioni: se $p$ passa i test è composito con probabilità (che consideriamo trascurabile) di $\frac{1}{2^30}$.

		Se $p$ passa il test è (molto probabilmente) primo: dobbiamo solo verificare se $q=2p+1$ lo è. Ripetiamo quindi il test di Solovay Strassen per $q$: se il risultato è nuovamente $q$ probabilmente primo, allora la generazione è finita e termina restituendo la coppia $p,q$.

		Tutto questo procedimento è riassunto nell'algoritmo \ref{algo:sophie_germain}.

		\begin{algorithm}
		\BlankLine
		\DontPrintSemicolon
	 	\caption{Test di Solovay Strassen}
	 	\label{algo:solovay_strassen}
		
		\balgo{$n$ numero da testare, $k$ numero di iterazioni}{\emph{false} se $n$ è composito, \emph{true} se è primo}

		\emph{isPrime} = \emph{true}\;
		\For {$i=1\rightarrow k$} {
			Scegli $a\in[2,n-1]$\;
			$x=\begin{pmatrix} a\\ n\end{pmatrix}$\;\label{row_l-j}
			\If{$x=0\lor a^{\frac{n-1} 2}\not\equiv x\mod n$} {\label{row_expmod}
				\emph{isPrime} = \emph{false}\;
			}
		}
		\Return(isPrime);
		\end{algorithm}

		Nel test di Solovay Strassen sono implicitamente utilizzati un algoritmo per calcolare il simbolo di Legendre-Jacobi (alla riga \ref{row_l-j}) e uno per l'esponenziazione modulare nella condizione a riga \ref{row_expmod} dell'algoritmo \ref{algo:solovay_strassen}.

		Per rendere il calcolo dell'esponenziazione modulare veloce si può usare l'algoritmo Square\&Multiply, riportato come algoritmo \ref{algo:square_multiply}.

		Per il simbolo di Legendre Jacobi usiamo invece l'algoritmo \ref{algo:legendre_jacobi}.

		\begin{algorithm}
		\BlankLine
		\DontPrintSemicolon
	 	\caption{Square\&Multiply}
	 	\label{algo:square_multiply}
		
		\balgo{}{}		
		\end{algorithm}

		\begin{algorithm}
		\BlankLine
		\DontPrintSemicolon
	 	\caption{Legendre Jacobi}
	 	\label{algo:legendre_jacobi}
		
		\balgo{$a<n\in\N$} {$\begin{pmatrix}a\\n\end{pmatrix}$}

		Da assicurarsi che sia corretto a mano\;
		%C'è qualche errore con la parte della condizione: perché valutiamo quel simbolo??
		\While{$a_i\neq1,2$} {
			Riduciamo $a_i \mod n_i$\;
			\uIf{$a_i$ pari} {
				Portiamo fuori tutte le potenze di 2\;
				Valutiamo $\begin{pmatrix}2\\n_i\end{pmatrix}$\;
			}
		Scambiamo $a_i, n_i$ con la reciprocità quadratica\;
		}
		\end{algorithm}

	\section{Crittosistema di Elgamal}
		Per la costruzione del crittosistema è necessario trovare un generatore in $\U(\Z_q)$. Usiamo il seguente teorema.
		\begin{prop}
			Se $p,q=2p+1$ sono una coppia di Sophie Germain, $x\in\Z_q$ è un generatore se e solo se $x^p\equiv -1\mod q$.
		\end{prop}
		Osserviamo che poiché $p, q$ sono primi, vale $x^p\mod q=\begin{pmatrix}x\\q\end{pmatrix}$.

Simuliamo una comunicazione e un attacco tramite l'algoritmo di Shanks: una classe costruirà il crittosistema vero e proprio, cercando un generatore modulo $p$ e calcolando il resto della chiave; poi un'altra classe cifrerà un messaggio e lo invierà tramite il crittosistema; infine un'ultima classe intercetterà il messaggio e cercherà di decifrarlo senza conoscere la chiave privata.
\end{document}
